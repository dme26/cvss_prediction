
\documentclass[11pt]{article}
\usepackage[english]{babel}
\usepackage[utf8]{inputenc}
\usepackage{fancyhdr}
\usepackage{enumitem}
\usepackage{array}
\usepackage{todonotes}
 
\pagestyle{fancy}
\fancyhf{}
\rhead{COSC480 Report} 

\lhead{Jake Norton (5695756)} 

\rfoot{\today}


\begin{document}

\noindent{\textsc{CVSS - Vulnerability Score Prediction}} \\
\noindent{
	Supervisor(s):
	David Eyers
	Veronica Liesaputra
}

\section{What are CVE and CVSS?}

\textit{The Common Vulnerabilities and Exposures (CVE) program is a dictionary or glossary of
	vulnerabilities that have been identified for specific code bases, such as software applications or
	open libraries.}[https://nvd.nist.gov/general/cve-process]

\subsection{Common Vulnerability Scoring System(CVSS)}

\textit{The Common Vulnerability Scoring System (CVSS) provides a way to capture the principal characteristics of a
	vulnerability and produce a numerical score reflecting its severity. The numerical score can then be translated
	into a qualitative representation (such as low, medium, high, and critical) to help organizations properly
	assess and prioritize their vulnerability management
	processes.}
%[https://www.first.org/cvss/#:~:text=The%20Common%20Vulnerability%20Scoring%20System,numerical%20score%20reflecting%20its%20severity.]

CVSS scoring is a high level way to break up vulnerabilities into different categories so that
organisations can choose which vulnerability to focus on first. CVSS in broken up into 3 distinct sections, base score,
temporal and environmental.

For brevity I will only show the specifics of CVSS 3.1 as this is by far the most commonly used version, even if it is
not the most recent.

\subsubsection*{Base Score}

\begin{itemize}

	\begin{itemize}

		\item Attack Vector -> Defines the avenues of attack that the vulnerability is open to. The more open a
		      component is, the higher the score. This can have the values Network, Adjacent, Local and Physical.

		\item Attack Complexity -> How complex the attack is so orchestrate. What are they prerequisites, how much
		      domain knowledge/ background work in necessary, how much effort does the attacker need to invest to
		      succeed. This can have the values Low or High. Low gives a higher base score.

		\item Priviledges Required -> The degree of priviledges the user needs to complete the attack. Generally
		      ranging from None, Low(e.g User level priviledge), High(e.g Administrator). The lower the priviledge
		      the higher the base score.

		\item User Interaction -> If the exploit requires another human user to make the attack possible, E.g
		      clicking a phishing link. This is either None or Required, the score is highest when no user
		      interaction is required.

		\item Scope -> Defines if the attack can bleed into other security scopes. E.g access to one machine gives
		      the ability to elevate privileges on other parts of the system. This can take Unchanged or Changed,
		      the score being highest when a scope change occurs.

		\item Confidentiality Impact -> Detemines what is the impact on the information access / disclosure to the
		      attacker. This can be High, Low or None with High adding the most to the base score.

		\item Integrity Impact -> Refers to the integrity of the information within the component. I.e could the
		      data have been modified by the attacker. This has High, Low or None as categories with High adding the
		      most to the base score.

		\item Availability Impact -> Refers to the impact of the attack on the availability of the component. E.g
		      the attacker taking the component off the network, denying the users access. This can haved High, Low
		      and None with High adding the most to the base score.

	\end{itemize}
\end{itemize}
%[https://www.first.org/cvss/v3.1/specification-document]
\subsubsection*{Temporal}

Temporal metrics describe the state of the exploit in relation to any further developments.

\bigskip

This could be:
\begin{itemize}

	\item Exploit Code Maturity -> The state of the attack itself, e.g has this exploit been pulled off in the wild or is it currently academic.

	\item Remidiation Level -> Broadly whether the exploit in question has been patched,

	\item Report Confidence -> The degree of confidence in the CVE report itself, the report may be in early stages where not all of the
	      information is known.

\end{itemize}

Temporal metrics would be useful in general for a CVSS score, however NVD do not store these temporal metrics. As far as
I can tell there is no reason given for this specifically, though discourse[stack exchange] around the subject suggests that this is due
to a lack of verifiable reporting. From my perspective both remidiation level and report confidence feel like they could
have scores attributed to them, however finding verifiable reports on the exploits seen in the wild does seem more
tricky, though there are two relatively new organisations on this front, CISA(public sector) and inthewild.org(private
sector).
%https://security.stackexchange.com/questions/270257/cvss-v3-and-v3-1-missing-temporal-metrics-exploit-code-maturity-and-remediation
%https://www.cisa.gov/known-exploited-vulnerabilities-catalog
%https://inthewild.io/

\subsubsection*{Environmental}

The environment metrics are there so that the specific user can modify the metrics to suit their specific circumstances,
the general idea is that you can scale a metric higher or lower, as such I will not go into any more detail here.

\section{Motive}
\subsection{Should we use CVSS?}
\todo[inline]{Read https://theoryof.predictable.software/articles/a-closer-look-at-cvss-scores/}

Wasn't originally designed to be used as a prioritization tool, however with the the massive rise in
the number of vulnerabilities, security analysts latched onto it as something they could use.
%[graph me daddy, https://cve.icu/CVEAll.html]
%https://www.tenable.com/blog/why-you-need-to-stop-using-cvss-for-vulnerability-prioritization
There have been a few attempts at dethroning CVSS. It does have some shortcomings:

\subsubsection*{Severity vs Risk}

The severity ideally is a measure of the impact(worst case? Or different levels?). Risk is the
likelihood of the event happening. However in CVSS this is a bit muddied as even the Base score
includes some aspects which defaults to worst case risk.

Should use some way of temporal / environmental. These are included in CVSS however, they are often
not used and it may be better to use EPSS, which gives a numerical value of the likelihood of the
event happening in 30 days based on previous history

These being the case, would the cyber sec community use a new standard if one came along?
Historically unlikely, especially given the uptake for CVSS 4.0 is so poor.

\subsubsection{Why 4.0 is not being used}
\todo[inline]{See if there are any valid reasons why it isn't instead of just inertia}
\todo[inline]{Honestly cannot find anything on the reasoning....}
\section{Exploration of the Space}

NVD is used by majority of studies as primary and sole data source.
Reasoning:
\begin{itemize}
	\item Data is in a nice format
	\item Largest collection of free CVE and CVSS scores, well maintained and in consistent format
\end{itemize}

Mitre is another large source.
Reasoning:
\begin{itemize}
	\item Format is more unwieldy
	\item Large lack of CVSS scores in comparison with NVD
\end{itemize}

Most other alternatives that were used in the Bayesian study have gone into archive status. There
are others however they are either pay to play VulnDB or they have a different focus, i.e
searchsploit.

\section{Analysis}

Analysis is useful in and of itself, knowing the breakdown of each metric, and the confidence there
in in relation to a prediciton will be useful. It can show shy the model may be less good at
prediction for certain metrics.

\subsection{Comparison with previous Bayesian study}

The previous analysis used bayesian techniques to compare and contrast different datasets. However,
those datasets were hard to attain, even then and were mostly webscraped. Additionally they were
using \textit{CVSS 2.0??}, where as we are now up to version 3.1.

\section{Results}
...


\subsection{Key take-aways}

Spending time with the dataset, finding its quirks, for example:

\begin{itemize}
	\item Mitre data only stores the vectors strings, but \~10 aren't in the expected format.
	\item Mitre has multiple entries for the same CVE with different scores, containing different
	      parameters. This is generally useful, but awkward when automating a training process.
\end{itemize}

\subsection{Is Mitre still useful to us?}

Yes, few possible use cases:

\begin{itemize}
	\item Use as the test data for the trained models.
\end{itemize}

This allows the model to be trained on all of NVD, but qureied on Mitre descriptions. This tests two
things... Its ability to predict based on unseen data(generalise), and an indirect similarity score.
Similarity score, insight into how different experts see the same problem.
Clustering? Look into which parts relate to what metric.



\paragraph{Aims}
\noindent
\end{document}
