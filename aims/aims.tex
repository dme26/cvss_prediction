
\documentclass[11pt]{article}
\usepackage[english]{babel}
\usepackage[utf8]{inputenc}
\usepackage{fancyhdr}
\usepackage{enumitem}
 
\pagestyle{fancy}
\fancyhf{}
\rhead{COSC480 Aims and Objectives} 

\lhead{Jake Norton (5695756)} 

\rfoot{\today}


\begin{document}

\noindent{\textsc{CVSS - Vulnerability Score Prediction}} \\
\noindent{
        Supervisor(s):
        David Eyers
        Veronica Liesaputra
}

\paragraph{Aims}
The primary aim of this research is to develop sophisticated predictive models capable of accurately determining
the severity levels of security threats based on the CVSS. This will involve a comprehensive review and comparison 
of current datasets, with a focus on leveraging natural language descriptions provided in security vulnerability reports. 
The project intends to utilize advanced transformer-based models to achieve this goal, contributing to the field of 
cybersecurity by enhancing the precision of threat severity assessments.

\paragraph{Objectives}
\begin{itemize}[noitemsep]
    \item Conduct a comprehensive literature review to understand the current landscape of CVSS score prediction and the methodologies employed in existing models.
    \item Replicate successful methodologies to verify the accuracy of CVSS score databases, with a particular focus on alignment with recent CVSS standards and datasets.
    \item Explore opportunities for enhancing existing methodologies, including the investigation of data amalgamation from multiple databases to ascertain improvements in model performance.
    \item Experiment with various model architectures to identify the most effective approach in terms of predictive accuracy, specifically focusing on metrics such as the F1 score and balanced accuracy.
\end{itemize}

\paragraph{Timeline} 
\begin{itemize}[noitemsep]
    \item March: Initiate the project with a literature review, system environment setup, and resource gathering.
    \item March-April: Replicate existing methodologies to validate findings and ensure alignment with current standards.
    \item May-June: Generate preliminary results and compile an interim report detailing findings and methodologies.
    \item July-August: Conduct experiments with various data source combinations and model architectures to identify optimal configurations.
    \item September-October: Finalize experimental work, analyze results, and prepare the comprehensive final report.
\end{itemize}

\noindent
\end{document}
